%yright 2007, 2008, 2009 Elsevier Ltd
%% 
%% This file is part of the 'Elsarticle Bundle'.
%% ---------------------------------------------
%% 
%% It may be distributed under the conditions of the LaTeX Project Public
%% License, either version 1.2 of this license or (at your option) any
%% later version.  The latest version of this license is in
%%    http://www.latex-project.org/lppl.txt
%% and version 1.2 or later is part of all distributions of LaTeX
%% version 1999/12/01 or later.
%% 
%% The list of all files belonging to the 'Elsarticle Bundle' is
%% given in the file `manifest.txt'.
%% 

%% Template article for Elsevier's document class `elsarticle'
%% with numbeblack style bibliographic references
%% SP 2008/03/01

% \documentclass[preprint,11pt]{elsarticle}
\documentclass[final,1p,11pt]{elsarticle}

%\documentclass[final,1p,times]{elsarticle}


%% Use the option review to obtain double line spacing
%%\documentclass[authoryear,preprint,review,12pt]{elsarticle}

%% Use the options 1p,twocolumn; 3p; 3p,twocolumn; 5p; or 5p,twocolumn
%% for a journal layout:
%% \documentclass[final,1p,times]{elsarticle}
%% \documentclass[final,1p,times,twocolumn]{elsarticle}
%% \documentclass[final,3p,times]{elsarticle}
%% \documentclass[final,3p,times,twocolumn]{elsarticle}
%% \documentclass[final,5p,times]{elsarticle}
%% \documentclass[final,5p,times,twocolumn]{elsarticle}

%%% For including figures, graphicx.sty has been loaded in
%% elsarticle.cls. If you prefer to use the old commands
%% please give \usepackage{epsfig}


\usepackage{epsfig}
%\usepackage{cite}
%\usepackage{mcite}
\usepackage{array,tabularx,epsfig,mathrsfs,graphicx,rotating}
\usepackage{ifthen}
\usepackage{amsfonts}
\usepackage{ragged2e}
\PassOptionsToPackage{hyphens}{url}
\usepackage[hyphens]{url}
\usepackage{hyperref}
\usepackage{listings}
\usepackage{lineno}
\usepackage{subfigure}
\usepackage{epstopdf}
% Custom colors
\usepackage{color}
\usepackage{float}
\usepackage{verbatim}
\usepackage{color,soul}
\usepackage{xcolor}
\usepackage{lipsum}
% to cross text
\usepackage[normalem]{ulem} % either use this (simple) or
\usepackage{soul} % use this (many fancier options)
\usepackage{amsmath,amssymb}

\let\originallesssim\lesssim
\let\originalgtrsim\gtrsim

\DeclareRobustCommand{\lesssim}{%
  \mathrel{\mathpalette\lowersim\originallesssim}%
}
\DeclareRobustCommand{\gtrsim}{%
  \mathrel{\mathpalette\lowersim\originalgtrsim}%
}

\makeatletter
\newcommand{\lowersim}[2]{%
  \sbox\z@{$#1<$}%
  \raisebox{-\dimexpr\height-\ht\z@}{$\m@th#1#2$}%
}
\makeatother


\hypersetup{
  colorlinks=true,
  linkcolor=black,
  citecolor=black,
  urlcolor=black
}




\graphicspath{{figs/}}


\pdfinfo{
   /Author (Chekanov et al)
   /Title  (Studies of granularity of a hadronic calorimeter for tens-of-TeV jets  at a 100 TeV pp collider)
   /CreationDate (D:2017)
   /Subject (PDFLaTeX)
   /Keywords (PDF;LaTeX)
}


\textheight=22cm
\textwidth=14.5cm

\newcommand{\beq}{\begin{equation}}
\newcommand{\eeq}{\end{equation}}
\newcommand{\la}{\langle}
\newcommand{\promc}{{\sc ProMC}}
\newcommand{\ra}{\rangle}
\newcommand{\eps}{\epsilon}
\newcommand{\ud}{\mathrm{d}}
\newcommand{\Ec}{\mathcal{E}}
\newcommand{\Fc}{\mathcal{F}}
\newcommand{\Za}{\mathrm{Z_1}}
\newcommand{\Zb}{\mathrm{Z_2}}
\newcommand{\Zn}{\mathrm{Z_n}}
\newcommand{\F}{\mathrm{F}}

\chardef\til=126
\newcommand{\GEANTfour} {\textsc{geant4}}
\newcommand{\pythia} {\textsc{Pythia8~}}
\newcommand{\pt}{\ensuremath{p_{\mathrm{T}}}}


\journal{}

\begin{document}
%\hfill ANL-HEP-149528
\definecolor{mygreen}{rgb}{0,0.6,0} \definecolor{mygray}{rgb}{0.5,0.5,0.5} \definecolor{mymauve}{rgb}{0.58,0,0.82}

\lstset{ %
 backgroundcolor=\color{white},   % choose the background color; you must add \usepackage{color} or \usepackage{xcolor}
 basicstyle=\footnotesize,        % the size of the fonts that are used for the code
 breakatwhitespace=false,         % sets if automatic breaks should only happen at whitespace
 breaklines=true,                 % sets automatic line breaking
 captionpos=b,                    % sets the caption-position to bottom
 commentstyle=\color{mygreen},    % comment style
 deletekeywords={...},            % if you want to delete keywords from the given language
 escapeinside={\%*}{*)},          % if you want to add LaTeX within your code
 extendedchars=true,              % lets you use non-ASCII characters; for 8-bits encodings only, does not work with UTF-8
 keepspaces=true,                 % keeps spaces in text, useful for keeping indentation of code (possibly needs columns=flexible)
 frame=tb,
 keywordstyle=\color{black},       % keyword style
 language=Python,                 % the language of the code
 otherkeywords={*,...},            % if you want to add more keywords to the set
 rulecolor=\color{black},         % if not set, the frame-color may be changed on line-breaks within not-black text (e.g. comments (green here))
 showspaces=false,                % show spaces everywhere adding particular underscores; it overrides 'showstringspaces'
 showstringspaces=false,          % underline spaces within strings only
 showtabs=false,                  % show tabs within strings adding particular underscores
 stepnumber=2,                    % the step between two line-numbers. If it's 1, each line will be numbeblack
 stringstyle=\color{mymauve},     % string literal style
 tabsize=2,                        % sets default tabsize to 2 spaces
 title=\lstname,                   % show the filename of files included with \lstinputlisting; also try caption instead of title
 numberstyle=\footnotesize,
 basicstyle=\small,
 basewidth={0.5em,0.5em}
}


\begin{frontmatter}

\title{
Return the comments for the referees on the report \\ JINST\_006P\_0219
}
%%%%%%%%%%%%%%%%%%%%%%%%%%%%%%%%%%%%%%%%%%%%%%%%%%%%%%%%%%%%%%%

\author[add3]{C.-H. Yeh}
\ead{a9510130375@gmail.com}

\author[add1]{S.V.~Chekanov}
\ead{chekanov@anl.gov}

\author[addDuke]{A.V.~Kotwal}
\ead{ashutosh.kotwal@duke.edu}

\author[add1]{J.~Proudfoot}
\ead{proudfoot@anl.gov}

\author[addDuke]{S.~Sen}
\ead{sourav.sen@duke.edu}

\author[add2]{N.V.~Tran}
\ead{ntran@fnal.gov}

\author[add3]{S.-S.~Yu}
\ead{syu@cern.ch}

\address[add3]{
Department of Physics and Center for High Energy and High Field Physics, 
National Central University, Chung-Li, Taoyuan City 32001, Taiwan
}

\address[add1]{
HEP Division, Argonne National Laboratory,
9700 S.~Cass Avenue,
Argonne, IL 60439, USA. 
}

\address[addDuke]{
Department of Physics, Duke University, USA
}

\address[add2]{
Fermi National Accelerator Laboratory
}




\begin{abstract}
Thanks for the comments from both referees, we did some minor revisions, and this document is used to describe the feedback and comments. 
\end{abstract}

\begin{keyword}
\end{keyword}

\end{frontmatter}

\section{The feedback and comments for referee 1}
Thank you for your encouragement and comments. For some points you mention in the report, describing as follows:\\
\begin{itemize}
%%%%%%%%%%%%%%%%%%%%%%%%%%%%%%%%%%%%%%%%%%%%%%%%%%%%%%%%%%%%
\item The note on the impact of HCAL granularity on jet-substructure variables at a future 100 TeV $pp$ collider is a very interesting read. It is well written and structublack and together with the single particle studies done in ref. [7] provides good evidence for investing in a hadron calorimeter with fine granularity.\\
\textcolor{red}{$\rightarrow$}Answer: Yes, thanks! This study is based on the same detector. 
%%%%%%%%%%%%%%%%%%%%%%%%%%%%%%%%%%%%%%%%%%%%%%%%%%%%%%%%%%%%
\item There are, however, a few shortcomings that impact the jet substructure more than the previously studied single particles: One such point is already raised in the concluding section 6: Apparently the blackuction of HCAL cell size from 5~$\times$~5 cm$^2$ to 1~$\times$~1 cm$^2$ does not improve the ROC curves anymore and in certain cases even worsens them for moderate signal efficiencies. This as mentioned in the conclusions is likely due to the details of the calorimeter clustering used prior to forming the jets studied here.\\
 \textcolor{red}{$\rightarrow$}Answer: Yes, we thought about the clustering issue and will investigate this in
the future. It should be noted that the current study uses the clustering parameters
as those for the SiD baseline detector, i.e. the clustering is already optimized for the
high-granularity case (1~$\times$~1~cm$^2$). 
\item The problem is that none of the relevant clustering parameters (size, dynamic growth, separation/merging criteria, etc.) are mentioned in the paper.\\
 \textcolor{red}{$\rightarrow$}Answer: We have added a sentence  in the Section "Simulation of detector response" and cited Refs [14] and [15] for the clustering parameters.\\ 
 \textcolor{red}{$\rightarrow$}Sentence: \textcolor{black}{The criteria for clustering in the calorimeter were discussed in Ref. [14]. We use the same criteria as those in the SiD detector design [15], which were optimized for a high-granularity HCAL with a cell size of 1~$\times$~1~cm$^2$.}\\
 %%%%%%%%%%%%%%%%%%%%%%%%%%%%%%%%%%%%%%%%%%%%%%%%%%%%%%%%%%%%
\item To my mind, the study should be extended by an investigation of clustering properties and their optimizations in the light of different granularity choices for the HCAL (and ECAL).\\
 \textcolor{red}{$\rightarrow$}Answer: Thanks. This is a preliminary study that uses the SiD-optimized clustering for the 1~$\times$~1 cm$^2$ case. The clustering was not optimized for the coarse granularity (5~$\times$~5 cm$^2$ and 20~$\times$~20 cm$^2$) considered in this paper. However, it is only natural to expect that optimization of clustering for other than 1~$\times$~1 cm$^2$ will make the conclusion in this paper even stronger. Given that any optimisation of clustering requires significant effort, such studies will be conducted in the future.\\
\item The second point is the source of correlated noise in the calorimeter cells in the form of showers stemming from additional $pp$ interactions observed together with the $pp$ collision of interest (so-called pile-up). The distinction made between signal and background in the paper is in fact between two different signal sources, $Z'\rightarrow q\bar{q}$ for background and either $Z' \rightarrow t\bar{t}$ or $Z' \rightarrow WW$) for signal. The jets formed by these do not suffer from the additional pp interactions, which in reality would be a major concern in reconstructing jet substructure reliably.\\
 \textcolor{red}{$\rightarrow$}Answer: Because we wanted to simplify the case and excluded the complicated jet conditions, we only focused on these three types of processes, including
QCD jets ($Z'\rightarrow q\bar{q}$), two-prong jets ($Z' \rightarrow WW$), and three-prong jets ($Z' \rightarrow t\bar{t}$).
It can help us to see the HCAL performance for different scenarios without 
the contamination from pileups. The condition mixed with pile-up could be our next step
for probing the jet performance of HCAL.
 %%%%%%%%%%%%%%%%%%%%%%%%%%%%%%%%%%%%%%%%%%%%%%%%%%%%%%%%%%%%
\item Still it is valuable to compare the jet substructure variables of these very high pT jets, given that the impact of pile-up would harm mostly lower energetic jets. I would like to encourage the authors to continue their studies along these two points (clustering and addition of pile-up). Despite the caveats just mentioned I find the results shown very encouraging in terms of highly granular hadron calorimeters and while the presented material is not the end of the story I recommend publishing the paper.\\
 \textcolor{red}{$\rightarrow$}Answer: Thanks! We will take these two points into account. They will be our next step for this research. 
 %%%%%%%%%%%%%%%%%%%%%%%%%%%%%%%%%%%%%%%%%%%%%%%%%%%%%%%%%%%%
\end{itemize}

\section{The feedback and comments for referee 2}
Thank you for your encouragement and comments. For some points you mention in the report, describing as follows:\\
\begin{itemize}
\item This paper presents studies for optimizing the hadronic calorimeter granularity to account for very collimated jets of very large energies.
The paper is in general clearly written and the result appears valid. I recommend the paper be published after the following, presumably minor, questions are satisfactorily addressed.\\
 \textcolor{red}{$\rightarrow$}Answer: Thanks! I will describe the revision as follows.\\
\item General questions: Is there a magnetic field assumed? Please state it in the text as that would help to open the collimated jets.\\
 \textcolor{red}{$\rightarrow$}Answer: Yes, it is based on the paper we published before, and we added a sentence in the section 2.\\
 \textcolor{red}{$\rightarrow$}Sentence: \textcolor{black}{The baseline detector discussed in Ref. [10]
includes a silicon-tungsten electromagnetic calorimeter with a transverse cell size of 2~$\times$~2~cm$^2$, a steel-scintillator hadronic calorimeter with a transverse cell size of 5~$\times$~5~cm$^2$, and a solenoid outside the ECAL and HCAL that provides a 5 T magnetic field.}
%%%%%%%%%%%%%%%%%%%%%%%%%%%%%%%%%%%%%%%%%%%%%%%%%%%%%%%%%%%%
\item References: The FCC/HE-LHC CDR have been released in January, I would then suggest adding/change when needed.\\
 \textcolor{red}{$\rightarrow$}Answer: We have added the reference in the section 1.\\
 \textcolor{red}{$\rightarrow$}Sentence: \textcolor{black}{Future circular $pp$ colliders [1] such as the European initiatives, FCC-ee [2], FCC-hh [3], high-energy LHC (HE-LHC) [4], and the Chinese initiative, SppC [5] }
\item Abstract: The values for the granularity given in the abstract are not motivated, especially the starting value. Because not only the eta/phi should matter, but also the longitudinal granularity, thus I would suggest to simply say with blackucing the cell size from a hadronic calorimeter by factor 4.\\
 \textcolor{red}{$\rightarrow$}Answer: We have modified the text in "abstract".\\
 \textcolor{red}{$\rightarrow$}Sentence: \textcolor{black}{.......with reducing cell size of a hadronic calorimeter 
from $\Delta \eta \times \Delta \phi = 0.087\times0.087$,
which are similar to the cell sizes of the calorimeters of LHC experiments, by a factor of four, to  $0.022\times0.022$.} 
 %%%%%%%%%%%%%%%%%%%%%%%%%%%%%%%%%%%%%%%%%%%%%%%%%%%%%%%%%%%%
\item 1.Introduction: Add the reference to HE-LHC CDR Volume 4, change the reference to FCC-hh CDR Volume 3\\
\textcolor{red}{$\rightarrow$}Answer: Thanks, the references have been added.
%%%%%%%%%%%%%%%%%%%%%%%%%%%%%%%%%%%%%%%%%%%%%%%%%%%%%%%%%%%%
\item 2.Simulation:\\
(1)It is not very clear from the text that you only change the granularity of the HCAL. I would suggest that you explicitly say the ECAL configuration you are using before talking about the HCAL and its segmentation.\\
 \textcolor{red}{$\rightarrow$}Answer: For question(1), we added the sentence 2-1 in the section 2.\\
 \textcolor{red}{$\rightarrow$}Sentence 2-1: \textcolor{black}{"The baseline detector discussed in Ref.[10]
includes a silicon-tungsten electromagnetic calorimeter with a transverse cell size of 2~$\times$~2~cm$^2$,......."} for ECAL configuration.\\
 \rule{\textwidth}{0.4pt}
(2)Add the reference to the Z model used and to Pythia8.\\
 \textcolor{red}{$\rightarrow$}Answer: For question (2), we added the references in the section 2 for the different processes in the sentence 2-2 and sentence 2-3 \\
\textcolor{red}{$\rightarrow$}Sentence 2-2:  \textcolor{black}{The $Z'$ bosons are forced to decay to two light-flavor quarks ($q\bar{q}$) [18], $W^+W^-$ [19] or $t\bar{t}$ [20] final states}.\\
\textcolor{red}{$\rightarrow$}Sentence 2-3: \textcolor{black}{\pythia generator [21].}\\
%%%%%%%%%%%%%%%%%%%%%%%%%%%%%%%%%%%%%%%%%%%%%%%%%%%%%%%%%%%%
\item 3. Studies of jet properties:\\
(1)not explained in the text that only $Z' \rightarrow WW$ are used here. \\
 \textcolor{red}{$\rightarrow$}Answer: For question (1), we added the sentence 3-1  in the section 3.\\
 \textcolor{red}{$\rightarrow$}Sentence 3-1: \textcolor{black}{........with the signal process $Z'\rightarrow WW$ only}.\\
 \rule{\textwidth}{0.4pt}
(2)I would not be so strict in the statement that cell sizes of LHC detectors are not suitable for tens of TeV jets. Indeed, it performs worse than better granularity, but still, the difference with lower granularity is not that large.\\ 
  \textcolor{red}{$\rightarrow$}Answer: For question(2), we added the sentence 3-2  to change the statement in the section 3 .\\
 \textcolor{red}{$\rightarrow$}Sentence 3-2: "\textcolor{black}{These experiments focus on jet substructure variables for jets
with $p_T \lesssim 4$~TeV. Our studies indicate that the future experiments, which will
measure jets with significantly greater transverse momenta, require an HCAL with higher granularity in order to achieve optimal performance for jet substructure variables. In the following sections we consider several other physics-motivated variables that can shed light on the performance of the HCAL for tens-of-TeV jets.}"\\
 \rule{\textwidth}{0.4pt}
(3)I would also like you to comment that for the largest mass considered, no difference is observed.\\
   \textcolor{red}{$\rightarrow$}Answer: For question(3), we added the sentence 3-3  in the section 3.\\
 \textcolor{red}{$\rightarrow$}Sentence 3-3: "\textcolor{black}{The extreme case with $M(Z')=40$ TeV corresponds to very boosted jets with 
 $p_T \simeq 20$~TeV.  This case does not show 
differences between the different HCAL configurations.}"\\
 %%%%%%%%%%%%%%%%%%%%%%%%%%%%%%%%%%%%%%%%%%%%%%%%%%%%%%%%%%%%
\item Caption of Fig 2 $\rightarrow$"granularities."\\
 \textcolor{red}{$\rightarrow$}Answer: Done, thanks!\\
\item 4.1 soft-drop: I would like some explanations on your choice of the two beta values for the study.\\
 \textcolor{red}{$\rightarrow$}Answer: we added a sentence in the section 4.1 to explain.\\
 \textcolor{red}{$\rightarrow$}Sentence: \textcolor{black}{For $\beta=0$ [33,34] the soft drop condition 
depends only on the $z_\mathrm{cut}$ and is angle-independent. At the parton level, this condition is infrared safe. For $\beta=2$ [35], the condition depends on both 
the angular distance between the two subjets and $z_\mathrm{cut}$, making the 
algorithm become both infrared and collinear safe at the parton level. Upon calorimeter clustering, the two $\beta$ values give different sensitivities to large-angle radiation.}\\
%%%%%%%%%%%%%%%%%%%%%%%%%%%%%%%%%%%%%%%%%%%%%%%%%%%%%%%%%%%%
\item 4.2 analysis method: Is the scan in mass done with the binning on Fig3,5,7,9? seems like you would benefit from smaller bins in the bulk of the distribution, but seems you might run out of statistics though.\\
 \textcolor{red}{$\rightarrow$}Answer: We added the sentence 4-2-1 in the Sect.4.2 to explain this problem.\\
 \textcolor{red}{$\rightarrow$}Sentence 4-2-1: \textcolor{black}{The ROC curves are computed with finely-binned histograms; the latter are rebinned coarsely for display purpose only.}\\
 \textcolor{red}{$\rightarrow$}Answer: We also add the sentence 4-2-2 in Sect.5 of the paper again.\\
 \textcolor{red}{$\rightarrow$}Sentence 4-2-2: \textcolor{black}{The ROC curves are computed with finely-binned histograms; the latter are rebinned coarsely for display purpose only.}
 %%%%%%%%%%%%%%%%%%%%%%%%%%%%%%%%%%%%%%%%%%%%%%%%%%%%%%%%%%%%
\item 4.3 results:\\
 (1)Label of figure 3,5,7,9 is not clear. Mention that the background is $Z'\rightarrow q\bar{q}$.\\
 \textcolor{red}{$\rightarrow$}Answer:  For question(1), we mention in the caption of the figures with the sentence 4-3-1.\\
 \textcolor{red}{$\rightarrow$}Sentence 4-3-1:  ......." The signal (background) process is $Z' \rightarrow WW$ ($Z'\rightarrow q\bar{q}$)....."\\
  \rule{\textwidth}{0.4pt}
\textcolor{black}{(2)Again, it would be good to comment that for the very high mass of 40 TeV, no difference can be seen.}\\
 \textcolor{red}{$\rightarrow$}Answer:  For question (2), the jets of the three types of processes are too boosted 
to be distinguished even with the smallest detector cell size. We have added this statement 4-3-2 and 4-3-3
in the results and conclusion of Section 4.3.\\
 \textcolor{red}{$\rightarrow$} \textcolor{black}{Statement 4-3-2: "Note that the separation between ROC curves depends on the physics variable and on the boost of the top quarks or the W bosons. For example, the similarity between the ROC curves shown in Fig. 6(a) is due to the insufficient boost of the top quarks, where even the largest cell size provides adequate discrimination from unstructured jets. On the other hand, Fig. 6(d) does not show a difference between the ROC curves because the boost is too high, where even the smallest cell size is not small enough, or the lateral spreading of the particle showers prevents discrimination from unstructured jets.  \\
 \textcolor{red}{$\rightarrow$} Statement 4-3-3: For both $Z' \to WW$ and $Z' \to t \bar{t}$ processes at $M(Z’) = 40$~TeV, the typical opening angle between the daughter jets 
 is 17 mrad or less; the smallest cell size we consider (1~$\times$~1~cm$^2$ or $\Delta \eta \times \Delta \phi = 4.3 \times 4.3$~mrad$^2$) 
 is not able to distinguish the substructure at this angular scale.}\\
 %%%%%%%%%%%%%%%%%%%%%%%%%%%%%%%%%%%%%%%%%%%%%%%%%%%%%%%%%%%%
\item 5.1 N-subjetiness: \\
(1)Same comment for the scan of Fig 11, 13, 15 as before, seems the binning is a bit too coarse for a scan.\\ 
 \textcolor{red}{$\rightarrow$}Answer: For question(1), same as before:  The ROC curves are computed with finely-binned histograms; the latter are rebinned coarsely for display purpose only.\\
 \rule{\textwidth}{0.4pt}
(2)For the scan, you add the bin with the larger number of SIGNAL events to extend the mass window? Sorry, now I'm confused by the end of page 11 and the beginning of page 13. Which scanning method is used in the end to make the ROC curves?\\ 
 \textcolor{red}{$\rightarrow$}Answer: For question(2). Sorry for the confusion, we deleted the "Same as Sect.4.2" and added a new sentence . This should avoid the confusion about which method we used.\\
 \textcolor{red}{$\rightarrow$}Sentence: \textcolor{black}{Following the suggestion of Ref. [38], the requirement on the 
soft drop mass with $\beta=0$ is applied before the study of $N$-subjettiness. 
For each detector configuration and resonance mass, the soft drop mass prerequisite window  
is determined as follows. The window is initialized by the median bin of the soft drop 
mass histogram from simulated signal events. Comparing the adjacent bins, the bin with the larger number of events is included to extend the mass window iteratively. The procedure is 
repeated until the prerequisite mass window cut reaches a signal  efficiency of 75\%.........} \\
 \rule{\textwidth}{0.4pt}
\textcolor{black}{(3)How can you interpret the fact that larger cell size gives better results for low Zmass?? Seems like something is not optimized, as there is no reason to see this behavior, and that will be good to comment more on it.}\\
 \textcolor{red}{$\rightarrow$}Answer: For question (3), at present, the reason for this effect is not understood because of the high complexity of the simulation of hadron shower for such jets  and complex reconstructions of clusters followed by reconstruction of jets. In fact, the reason why we run such complex simulation is to give the answers to such questions. Note that the cluster reconstruction is already optimized to 1~$\times$~1~cm$^2$ cells (since we use the SiD setup). The simulation may not be fully optimized for large cell sizes. This means that our conclusion can only be stronger after better optimisation of clustering for larger cell sizes. \\ 
\end{itemize}
\end{document}
